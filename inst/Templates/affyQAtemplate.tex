\documentclass[11pt]{article}

\usepackage{times}
\usepackage{graphicx}
\usepackage{hyperref}

\newcommand{\Rfunction}[1]{{\texttt{#1}}}
\newcommand{\Robject}[1]{{\texttt{#1}}}
\newcommand{\Rpackage}[1]{{\textsf{#1}}}
\newcommand{\Rmethod}[1]{{\texttt{#1}}}
\newcommand{\Rfunarg}[1]{{\texttt{#1}}}
\newcommand{\Rclass}[1]{{\textit{#1}}}

\begin{document}

\title{Quality Report for Affymetrix Microarray Experiments}

\maketitle

This is a QA report for the dataset @repName@.
For details on packages used see Section~\ref{sec:ack}.

\section{Affymetrix recommended QA}

In this section different statistics based on Affymetrix recommendations
are reported. These statistics were generated using the \Rpackage{simpleaffy} 
package. If you are not familiar with the diagnostics reported here we
strongly recommend consulting the documentation and vignettes in the
\Rpackage{simpleaffy} package.

@TABLE1@


In Table~\ref{table1} both the average background and the scale 
factors, for each array, are reported.  The average background values should 
be similar to each other, across arrays. Scale factors should be within
a factor of 3 of each other.

@TABLE2@

In Table~\ref{table2} ratios of hybridization efficiency, between 3' and
5' situated probes sets are presented. These should all be less than 3.

These results are also summarized in Figure~\ref{fig:sA}. Any QA statistic
that is shown in red is out of the manufacturer's specified boundaries
and suggests a potential problem.


\begin{figure}[tp]
  \centering
\includegraphics{@sA@}

\caption{\label{fig:sA}%
The \Rpackage{simpleaffy} QC diagnostic plot.}
\end{figure}

\section{QA procedures from the \Rpackage{affy} package}

The next series of plots were constructed using software from the 
\Rpackage{affy} package. Readers unfamiliar with them and their
interpretation are referred to the documentation and vignettes in
that package.

Figure~\ref{fig:hist} shows histograms (density estimates) of the raw, 
per probe expression values, for each chip. Figure~\ref{fig:bxp}
presents boxplots of the same data.  Note these are computed pre-background
correction, normalization and summarization and represent the raw
data. Arrays that look dramatically different from the others should be 
considered for possible problems, especially if they also appear in any
of the other diagnostics in this report.

\begin{figure}[tp]
  \centering
\includegraphics{@hist@}
\caption{\label{fig:hist}%
affy histograms.}
\end{figure}

\begin{figure}[tp]
  \centering
\includegraphics{@bxp@}
\caption{\label{fig:bxp}%
affy boxplot.}
\end{figure}


In Figure~\ref{fig:rnadeg} a RNA digestion plot is computed. In this plot
each array is represented by a single line. It is important to identify 
any array(s) that has a slope which is very different from the others. 
The indication is that the RNA used for that array has potentially 
been handled quite differently from the other arrays. 

\begin{figure}[tp]
  \centering
\includegraphics{@RNAdeg@}
\caption{\label{fig:rnadeg}%
RNA digestion/degredation plot.}
\end{figure}

\section{QA tools from the \Rpackage{affyPLM} package}

In this section we present diagnostic plots based on tools provided
in the \Rpackage{affyPLM} package.

\begin{figure}[tp]
  \centering
\includegraphics{@NUSE@}
\caption{\label{fig:NUSE}%
NUSE plot.}
\end{figure}

\begin{figure}[tp]
  \centering
\includegraphics{@RLE@}
\caption{\label{fig:RLE}%
RLE plot.}
\end{figure}

\section*{Acknowledgements}
\label{sec:ack}

This report was generated using version @affyQCVersNO@ of the 
\Rpackage{affyQCReport}, written by Craig Parman, Conrad Halling, and
Robert Gentleman. It uses functions from the \Rpackage{affy} package
written by R. Irizarry et al, the \Rpackage{simpleaffy} package, written
by C. J. Miller and the \Rpackage{affyPLM} package written by B. M. Bolstad.
W. Huber contributed substantially to the format and functions.

SessionInformation: 
@sessionInfo@

\end{document}
