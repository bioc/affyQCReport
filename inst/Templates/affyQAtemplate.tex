\documentclass[11pt]{article}

\usepackage{times}
\usepackage{graphicx}
\usepackage{hyperref}

\newcommand{\Rfunction}[1]{{\texttt{#1}}}
\newcommand{\Robject}[1]{{\texttt{#1}}}
\newcommand{\Rpackage}[1]{{\textsf{#1}}}
\newcommand{\Rmethod}[1]{{\texttt{#1}}}
\newcommand{\Rfunarg}[1]{{\texttt{#1}}}
\newcommand{\Rclass}[1]{{\textit{#1}}}

\begin{document}

\title{Quality Report for Affymetrix Microarray Experiments}

\maketitle

This is a QA report for the dataset @repName@.

@TABLE1@

In Table 1 we see both the average background and the scale factors for each
array.  The average background values should be similar to each other, 
across arrays, as should the scale factors. Scale factors should be within
a factor of 3 of each other.


@TABLE2@




\begin{figure}[tp]
  \centering
\includegraphics{@sA@}

\caption{\label{fig:sA}%
simpleaffy QC diagnostic plot.}
\end{figure}

\begin{figure}[tp]
  \centering
\includegraphics{@hist@}
\caption{\label{fig:hist}%
affy histograms.}
\end{figure}

\begin{figure}[tp]
  \centering
\includegraphics{@bxp@}
\caption{\label{fig:bxp}%
affy boxplot.}
\end{figure}


\begin{figure}[tp]
  \centering
\includegraphics{@RNAdeg@}
\caption{\label{fig:rnadeg}%
RNA digestion/degredation plot.}
\end{figure}

\begin{figure}[tp]
  \centering
\includegraphics{@NUSE@}
\caption{\label{fig:NUSE}%
NUSE plot.}
\end{figure}

\begin{figure}[tp]
  \centering
\includegraphics{@RLE@}
\caption{\label{fig:RLE}%
RLE plot.}
\end{figure}

\section{Acknowledgements}

This report was generated using version @affyQCVersNO@ of the 
\Rpackage{affyQCReport}, written by Craig Parman, Conrad Halling, and
Robert Gentleman. It uses functions from the \Rpackage{affy} package
written by R. Irizarry et al, the \Rpackage{simpleaffy} package, written
by C. J. Miller and the \Rpackage{affyPLM} package written by B. M. Bolstad.
W. Huber contributed substantially to the format and functions.

SessionInformation: 
@sessionInfo@

\end{document}
